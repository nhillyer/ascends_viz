\documentclass[12pt]{article}

\usepackage{palatino}
\usepackage{amsmath}
\usepackage{graphicx}

% Idiopidae
\usepackage{fancybox}
\usepackage[]{geometry}
\usepackage{fancyvrb}
\usepackage{color}
\newcommand\at{@}
\newcommand\lb{[}
\newcommand\rb{]}
\newcommand\PYba[1]{\textcolor[rgb]{0.00,0.00,0.50}{\textbf{#1}}}
\newcommand\PYaz[1]{\textcolor[rgb]{0.00,0.63,0.00}{#1}}
\newcommand\PYay[1]{\textcolor[rgb]{0.67,0.13,1.00}{\textbf{#1}}}
\newcommand\PYax[1]{\textcolor[rgb]{0.60,0.60,0.60}{\textbf{#1}}}
\newcommand\PYbc[1]{\textcolor[rgb]{0.67,0.13,1.00}{\textbf{#1}}}
\newcommand\PYas[1]{\textcolor[rgb]{0.73,0.27,0.27}{\textit{#1}}}
\newcommand\PYar[1]{\textcolor[rgb]{0.72,0.53,0.04}{#1}}
\newcommand\PYaq[1]{\textcolor[rgb]{0.53,0.00,0.00}{#1}}
\newcommand\PYap[1]{\textcolor[rgb]{0.00,0.50,0.00}{#1}}
\newcommand\PYaw[1]{\textcolor[rgb]{0.40,0.40,0.40}{#1}}
\newcommand\PYav[1]{\textcolor[rgb]{0.67,0.13,1.00}{\textbf{#1}}}
\newcommand\PYau[1]{\textcolor[rgb]{0.40,0.40,0.40}{#1}}
\newcommand\PYat[1]{\textcolor[rgb]{0.72,0.53,0.04}{#1}}
\newcommand\PYak[1]{\textcolor[rgb]{0.73,0.40,0.53}{#1}}
\newcommand\PYaj[1]{\textcolor[rgb]{0.67,0.13,1.00}{#1}}
\newcommand\PYai[1]{\textcolor[rgb]{0.72,0.53,0.04}{#1}}
\newcommand\PYah[1]{\textcolor[rgb]{0.63,0.63,0.00}{#1}}
\newcommand\PYao[1]{\textcolor[rgb]{0.73,0.40,0.13}{\textbf{#1}}}
\newcommand\PYan[1]{\textcolor[rgb]{0.67,0.13,1.00}{\textbf{#1}}}
\newcommand\PYam[1]{\textbf{#1}}
\newcommand\PYal[1]{\textcolor[rgb]{0.00,0.53,0.00}{\textbf{#1}}}
\newcommand\PYac[1]{\textcolor[rgb]{0.73,0.27,0.27}{#1}}
\newcommand\PYab[1]{\textit{#1}}
\newcommand\PYaa[1]{\textcolor[rgb]{0.50,0.50,0.50}{#1}}
\newcommand\PYag[1]{\textcolor[rgb]{0.40,0.40,0.40}{#1}}
\newcommand\PYaf[1]{\textcolor[rgb]{0.00,0.53,0.00}{\textit{#1}}}
\newcommand\PYae[1]{\textcolor[rgb]{0.40,0.40,0.40}{#1}}
\newcommand\PYad[1]{\textcolor[rgb]{0.73,0.27,0.27}{#1}}
\newcommand\PYbb[1]{\textcolor[rgb]{0.40,0.40,0.40}{#1}}
\newcommand\PYaZ[1]{\textcolor[rgb]{0.00,0.50,0.00}{\textbf{#1}}}
\newcommand\PYaY[1]{\textcolor[rgb]{0.73,0.27,0.27}{#1}}
\newcommand\PYaX[1]{\textcolor[rgb]{0.67,0.13,1.00}{#1}}
\newcommand\PYbf[1]{\textcolor[rgb]{0.73,0.40,0.53}{\textbf{#1}}}
\newcommand\PYbg[1]{\textcolor[rgb]{0.67,0.13,1.00}{\textbf{#1}}}
\newcommand\PYbd[1]{\textcolor[rgb]{0.73,0.27,0.27}{#1}}
\newcommand\PYbe[1]{\textcolor[rgb]{0.40,0.40,0.40}{#1}}
\newcommand\PYaS[1]{\textcolor[rgb]{0.00,0.53,0.00}{\textit{#1}}}
\newcommand\PYaR[1]{\textcolor[rgb]{0.72,0.53,0.04}{#1}}
\newcommand\PYaQ[1]{\textcolor[rgb]{0.40,0.40,0.40}{#1}}
\newcommand\PYaP[1]{\textcolor[rgb]{0.73,0.27,0.27}{#1}}
\newcommand\PYaW[1]{\textcolor[rgb]{0.73,0.27,0.27}{#1}}
\newcommand\PYaV[1]{\textcolor[rgb]{0.00,0.00,1.00}{\textbf{#1}}}
\newcommand\PYaU[1]{\textcolor[rgb]{0.82,0.25,0.23}{\textbf{#1}}}
\newcommand\PYaT[1]{\textcolor[rgb]{0.50,0.00,0.50}{\textbf{#1}}}
\newcommand\PYaK[1]{\textcolor[rgb]{0.00,0.63,0.00}{#1}}
\newcommand\PYaJ[1]{\textcolor[rgb]{0.73,0.27,0.27}{#1}}
\newcommand\PYaI[1]{\textcolor[rgb]{0.00,0.73,0.00}{\textbf{#1}}}
\newcommand\PYaH[1]{\fcolorbox[rgb]{1.00,0.00,0.00}{1,1,1}{#1}}
\newcommand\PYaO[1]{\textcolor[rgb]{0.00,0.00,0.50}{\textbf{#1}}}
\newcommand\PYaN[1]{\textcolor[rgb]{0.00,0.00,1.00}{#1}}
\newcommand\PYaM[1]{\textcolor[rgb]{0.00,0.53,0.00}{#1}}
\newcommand\PYaL[1]{\textcolor[rgb]{0.73,0.73,0.73}{#1}}
\newcommand\PYaC[1]{\textcolor[rgb]{0.67,0.13,1.00}{#1}}
\newcommand\PYaB[1]{\textcolor[rgb]{0.00,0.25,0.82}{#1}}
\newcommand\PYaA[1]{\textcolor[rgb]{0.67,0.13,1.00}{#1}}
\newcommand\PYaG[1]{\textcolor[rgb]{0.72,0.53,0.04}{#1}}
\newcommand\PYaF[1]{\textcolor[rgb]{1.00,0.00,0.00}{#1}}
\newcommand\PYaE[1]{\textcolor[rgb]{0.63,0.00,0.00}{#1}}
\newcommand\PYaD[1]{\textcolor[rgb]{0.00,0.53,0.00}{\textit{#1}}}

\newcommand{\versal}[1]{\noindent{\Huge #1\kern-.10em}}
\newcommand{\file}[1]{{\ttfamily\bf #1}}
\newcommand{\ident}[1]{{\ttfamily\it #1}}
\newcommand{\shell}[1]{{\ttfamily\it #1}}
\newcommand{\python}[1]{{\ttfamily\it #1}}

\newenvironment{code}[2]
    {
        \begin{center}
        \label{#1}\ovalbox{
            \begin{minipage}{.8\textwidth}
            \begin{center}
            {\bf Code: #1} -- {\it #2}
            \end{center}
            \end{minipage}
        }
        \end{center}
        \sffamily\small
    }
    {\normalsize\rmfamily\marginpar{$\hookleftarrow$}}


\begin{document}
\author{Benjamin S. Hughes}
\title{Confessions of the ASCENDS Data Visualization Team}
\maketitle

This document attempts to justify and explain the methods to our madness.  In this tome, the structure and interpretation of the data we have been given shall be divined and documented.  The mathematical incantations will be demystified.  Basically, this document will change your life.

\section*{Data}

The data we received came in two categories: insitu and ITT.  The insitu data are the mixing ratio (in ppm) as recorded at the altitude of the airplane. The ITT data are the value of the backscatter of a laser that is absorbed by CO$_2$. This value can then be used to calculate the number density of CO$_2$ in the column of air below the laser.  The ITT data are the primary focus of the experiment.

The data are organized into flights (which makes sense because that is how the data are gathered).  Flights are named as \texttt{\%m\%d\%y\_flightN/} where \texttt{N} is the flight number.  In the original file structure there are the folders \texttt{insitu/\%m\%d\%y\_flightN/} and \texttt{itt/\%m\%d\%y\_flightN/}; for processing, we change the directory structure to \texttt{\%m\%d\%y\_flightN/insitu/} and \texttt{\%m\%d\%y\_flightN/itt/}.

\subsection*{Insitu}

The insitu data are in CSV files. The CO$_2$ readings are found in the files \texttt{lear*.txt}.  The first column seems to be nonsensical and a constant 999.  The second column is seconds since midnight (in localtime/EDT).  The third column is the mixing ration (parts per million) of CO$_2$ at the altitude the plane was flying.

\subsection*{ITT}

The ITT data are spread over a variety of files and formats.  The two that are of interest are the files that contain the CO$_2$ readings which match the pattern \texttt{*.dbl} and the GPS information which is contained in the file \texttt{*in-situ\_gps\_serial \_data.txt}.

The CO$_2$ readings are stored as a binary file of 8-byte floating point values (doubles) in little endian form.  The first value is the number of seconds since January 1, 1904 00:00 GMT \footnote{Yes, it is different from the UNIX epoch of 1970-01-01 00:00 GMT, conversion code is necessary}. Measurements are taken at 5Hz. The format is given in the table:

\begin{center}
\begin{tabular}{|l|l|l|l|l|l|l|l|l|}
\hline 
Time & Ref\_On & Ref\_Side & Ref\_Off & Blank & Sci\_On & Sci\_Side & Sci\_Off & Blank \\
\hline
\end{tabular}
\end{center}

The GPS positions are in a CSV file.  The timestamp is the first two elements on a line and is in the form \texttt{\%m/\%d/\%Y \%H:\%M:\%S}.  The longitude and latitude are stored in elements nine through twelve.  They are stored as hemisphere character and then an integer.  \texttt{DataPoint::make\_lat\_lon} converts them to floating points by splitting up the string of the number and converting into degrees, minutes, and seconds. The longitude value seems to have an extraneous 0 prepended that is removed during processing.

During the import phase, all values are combined and averaged for each second.  This allows the GPS and CO$_2$ data to be merged easily.

ITT CO$_2$ data is taken from the side sensor and is stored as the ratio of Sci\_Side and Ref\_Side.  

\section*{Visualizations}

\subsection*{Geographic Geometry}

The visualizations work because of maths.  Maths is the study of numbers.  The earth is a sphere\footnote{No, it actually isn't, but in our world it is}.  For these visualizations, we use spherical trigonometry to find latitude and longitude coordinates.  We have encapsulated many common operations into \texttt{KmlTools}.

Often it is necessary to find the heading (clockwise angle originating from North) between two points.  This can be found with the following process.

\begin{equation}
\begin{aligned}
A &=\Delta \lambda \\
c &=90^\circ - \phi_2 \\
b &=90^\circ - \phi_1 \\
\end{aligned}
\end{equation}

\begin{equation}
a = \arccos{(\cos{b}\cos{c} + \sin{b}\sin{c}\cos{A})}
\end{equation}

\begin{equation}
C =\arcsin{\left( \frac{\sin{A}\sin{c}}{\sin{a}} \right)}
\end{equation}

\begin{equation}
B = 180^\circ - (A + C)
\end{equation}

\begin{equation}
h(x) = \begin{cases}
  360^\circ - C, & \phi_1<\phi_2 \\
  180^\circ - C, & \phi_1>\phi_2 \\
\end{cases}
\end{equation}

It is also useful to be able to find the distance between two points.  This can be found using an adaptation of the Haversine Formula.

\begin{equation}
\text{Earth}_\text{radius}\cdot 2\arcsin\left(\sin^2{\left(\frac{\Delta \phi}{2}\right)} + \cos(\phi_1)\cos(\phi_2)\sin^2\left(\frac{\Delta \lambda}{2}\right)\right)
\end{equation}

Going in the opposite direction, coordinates of the form ($\phi_2$,$\lambda_2$) can be found at a distance $d$ and an angle (heading) $h$ by:


\begin{equation}
\label{great-circle-coords-lat}
\phi_2 = \arcsin(\sin\phi_1\cos{d}+\cos\phi_1\sin d\cos(h))
\end{equation}

\begin{equation}
\label{great-circle-coords-dlon}
\Delta\lambda = \arctan\left(\frac{\sin(h)\sin d\cos\phi_1}{\cos d-\sin\phi_1\sin\phi_2}\right)
\end{equation}

\begin{equation}
\lambda_2 = \left((\lambda_1+\Delta\lambda+\pi) \mod 2\pi\right)-\pi
\end{equation}

Equation (\ref{great-circle-coords-lat}) is derived from the spherical law of cosines

\begin{equation}
\cos{b} =\cos{c}\cos{a} + \sin c\sin a\cos B
\end{equation}

where $a=\frac{\pi}{2}-\phi_1$, $b=\frac{\pi}{2}-\phi_2$, $c=d$, and $B=h$.  The derivation of Equation (\ref{great-circle-coords-dlon}) is much more involved and beyond the scope of this document.

\texttt{KmlTools} also contains applications of these formul\ae{} for the drawing of shapes.  These shape functions often come in two forms, \textit{shape}() and \textit{shape}\texttt{\_coords}().  The \texttt{\_coords} function just generates the \texttt{[lon,lat,alt]} coordinates, while the undecorated name wraps the coordinates into a \texttt{KML::Polygon}.

\subsection*{lisp.rb}

Due to personal preference, I have included a small library file called \texttt{lisp.rb}.  It includes the functions \texttt{car} and \texttt{cdr}.  They both take a single argument, an array.  \texttt{car} returns the first element of the array (the head) and \texttt{cdr} returns all the elements after the first (the tail).  The appearance of these functions throughout the code might be unsettling at first, but they are truly helpful to me.

\subsection*{Tasks}

The visualizations are organized into Rake tasks.  Rake is a standard ruby utility that provides Make-like functionality.  These tasks can be run from the command line with \texttt{rake task\_name}.  This provides a convenient way to manage the various capabilities of this application.

\subsubsection*{import}

Import loads flight data from \texttt{input/} or the path specified by the \texttt{INPUT\_PATH} variable on the command line.  

Flight data is imported with the function \texttt{Flight::load()}. This function creates a new Flight record, extracting the flight date and flight number from the directory name.  It then delegates out to \texttt{DataPoint::from\_files()}.

\begin{code}{from-files1}{Merging Tuples}
### @include "../models/data_point.rb" "Merge Tuples"
### @end
\end{code}

From\_files works by generating tuples of the form \texttt{[unix time, ...]} from the ITT data and the insitu data. These tuples are transformed into hashes with the timestamp as the key.  This allows them to be easily merged for a given second.  Only datapoints with ITT data are of interest, so all tuples with fewer than 10 elements are discarded.

The DataPoints are then created for the tuples.

\begin{code}{from-files2}{Creating DataPoints}
### @include "../models/data_point.rb" "Create DataPoints"
### @end
\end{code}

ITT data is read from the \texttt{*.dbl} files in the \texttt{itt/} directory.  Each file is read as a series of records consisting of 9 8-byte double-precisionnumbers in little-endian byte format.  These are unpacked into an array.  The data\_points tuple is constructed by trimming the dat array.  The timestamps are taken from 1904, so the \texttt{EPOCH\_FAIL} (bad pun) constant is subtracted to put them into standard unix time.  The data\_points are then averaged to the nearest second from the sample rate of 5Hz.

\begin{code}{from-itt-data}{From ITT DataPoints}
### @include "../models/data_point.rb" "From ITT DataPoints"
### @end
\end{code}

The process for reading the GPS data is very similar, except the data are comma delimited.  The GPS and ITT data are merged by timestamp into the tuples that are returned.

The reading of the GPS data and the ITT data take place in parallel, thanks to JRuby's real threads.


Insitu data is read from the \texttt{lear*.txt} CSV.  The timestamp is in seconds since the midnight of the flight date in EDT (-0400).  Fill values are discarded.

\begin{code}{from-insitu}{From Insitu}
### @include "../models/data_point.rb" "From Insitu"
### @end
\end{code}

\subsubsection*{plot\_flightpaths}

Plot\_flightpaths assembles the kml file \texttt{output/datapoint\_paths.kml} which contains the flightpaths of all the flights in the database.  The flightpaths are generated from the DataPoint coordinates and represented by a \texttt{KML::LineString}.

\subsubsection*{plot\_co2\_columns}

This task creates color coded columns corresponding to the values of the columns of air underneath the flightpath.  On top of these columns, boxes are drawn that correspond to the ppm measurement of the insitu instrument at the flight altitude.  This visual is output to \texttt{output/co2\_columns.kml}.

\begin{code}{plot-co2-columns}{Create CO2 Columns}
### @include "../tasks/plot_co2_columns.rake" "Create CO2 Columns"
### @end
\end{code}

The process for creating the visuals for Google Earth is two-step.  First column coordinates are created by calculating the headings between each adjacent datapoint.  Points are then found at a radius of 150m that form a line perpendicular to the heading and intersect the first of the datapoints.

These column coordinates are then handled by \texttt{make\_column\_placemarks}, where the coordinates are used to create squares extruded to the ground.  KML style attributes are used in conjunction with the \texttt{CO2ColorCode} to apply color to the columns. Finally, the columns are encapsulated within a placemark for display in Google Earth.  

The InSitu boxes are drawn on top using a similar method.  A merging algorithm combines similarly colored adjacent boxes to help reduce file sizes.  It seems that this can often reduce the file size by a factor of 2.

\subsubsection*{make\_color\_bar}

This task creates a color bar for the CO$_2$ insitu and ITT data.  The output is a PNG image at \texttt{output/co2\_color\_bar.png} and a KML file which overlays the image on the screen at \texttt{output/co2\_color\_bar.kml}.  The color bar should be the same as Figure (\ref{co2-color-bar}).   

\begin{figure}
\centering
\includegraphics[width=19mm]{../output/co2_color_bar.eps}
\caption{Color Bar for CO$_2$ data.}
\label{co2-color-bar}
\end{figure}

This is generated using Ruby-Processing in Co2ColorCodeBar and Co2ColorCode classes.  The drawing takes place in Co2ColorCodeBar and is pretty standard Processing code.  

The color coding takes place in Co2ColorCode and applies a function to a normalized value (on range [0.0, 1.0]).  Insitu and ITT values are normalized to this range through global variables that define the upper and lower bounds for the expected sensor values.

\begin{code}{color-bounds}{CO2 Bounds Variables}
### @include "../lib/co2_color_code.rb" "Bounds Variables"
### @end
\end{code}

ITT values are colored through \texttt{Co2ColorCode::itt\_colorify} and Insitu values are colored through \texttt{Co2ColorCode::insitu\_colorify}.  Once the values are normalized they are passed to the function \texttt{normalized\_colorify} which assigns colors in the form of $(r,g,b)$ (where each is on the range [0,255]) through the function

\begin{equation}
\label{normalized-colorify}
c(x)= \begin{cases}
(255,255,255), & x < \frac{1}{27} \\
(-918x+289,-783x+169,255), & x < \frac{6}{27} \\
(-1215x+448,-1377x+561,255), & x < \frac{11}{27} \\
(-1296x+719,255,-1296x+719), & x < \frac{16}{27} \\
(255,-1296x+1023,0), & x < \frac{22}{27} \\
(-1026x+1052,0,432x-337), & x < \frac{26}{27} \\
(0,0,0), & x >= \frac{26}{27} \\
\end{cases}
\end{equation}

These values are then transformed into hexadecimal and assembled into an 8 character string in the form ``AABBGGRR''.  The alpha channel is set to a constant 255 for full opacity.

\subsubsection*{run\_hysplit}

This task creates the configuration files necessary to run the HYSPLIT program.  It then runs HYSPLIT for data points at a periodic interval.  It requires one to be a registered user of HYSPLIT and have it installed to \texttt{/hysplit4/}.

\subsubsection*{plot\_hysplit}

This task uses the standard hysplit utilities to create KML for the HYSPLIT output generated from \texttt{run\_hysplit}.  The output is placed in \texttt{output/ hysplit/kml/}

\subsubsection*{merge\_hysplit\_paths}

This task merges all the hysplit plots for each flight into a single, combined kml of the form \texttt{output/hysplit/kml/merged/flight\_N.kml}.

\subsubsection*{import\_emitters}

This imports possible point sources of CO$_2$ from the file specified by the \texttt{FILE} variable.  The format is CSV with the values \texttt{name, lat, lon, energy source}.

\subsubsection*{plot\_emitters}

This task plots the emitters in Google Earth.  It uses cute 3d models of power plants to mark the emitters on the map. The output is \texttt{output/emitters.kmz}

\subsubsection*{complete\_visualization}

This task combines all the visualization products into one kml file.  CO$_2$ columns and HYSPLIT plots are merged based on time.  HYSPLIT is hidden by default.  The emitters and the color bar are also included.  The output is \texttt{output/complete\_vi sualization.kml}.  This file depends upon \texttt{emitters.kmz} and \texttt{co2\_color\_bar.png}.

\end{document}
